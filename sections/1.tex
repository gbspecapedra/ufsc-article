\section{Introdução} \label{cap:introducao}

A música era um meio de comunicação exclusivamente presencial. No decorrer do tempo, as técnicas e invenções aplicadas ao processo de gravação do som foram surgindo e se aperfeiçoando. Com a Internet, a música ultrapassa os limites físicos da mídia, mergulhando no universo digital. Desta forma, a organização da informação, que inclui a sua representação, tem a principal finalidade de possibilitar a recuperação dessa informação, além da sua guarda para a posteridade. A música se tornou um objeto de consumo universal e extremamente acessível \cite{daquino2012, gomes2015}.

A busca por similaridade musical está inserida dentro de um tema de estudos denominado \textit{Music Information Retrieval}. Os pesquisadores de MIR observam que a motivação maior para essa área de pesquisa é o grande volume de música digital disponível na Internet que, quanto mais cresce, menos possibilita sua recuperação eficiente, visto que estão disponíveis em grande volume, mas sem o tratamento adequado \cite{mclane1996, downie2003, santinisouza2007}.

Este trabalho tem como objetivo geral estudar o estado da arte sobre a recuperação da informação de dados musicais e realizar uma análise comparativa de algumas soluções existentes que não necessariamente buscam dados musicais apenas através do casamento direto de parâmetros de entrada para a busca, mas também através da similaridade destes parâmetros.

Para atingir este objetivo, foi necessário entender os conceitos básicos de som e como o som é transformado em áudio digital (\cite{muller2007, paulozuben2004, pachecolopes2014}); identificar os formatos de dados musicais e como é feito o armazenamento deles em bancos de dados ou repositórios digitais (\cite{rafael2013, michels1992}); e estudar os métodos e algoritmos utilizados para busca por similaridade de dados musicais.

Portanto, a relevância deste trabalho pretende contribuir diretamente, para agregar conhecimento, com o estudo sobre a recuperação da informação de dados musicais, que auxiliarão no desenvolvimento futuro de soluções para busca por similaridade de dados musicais.
