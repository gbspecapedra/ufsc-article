\section{Análise Comparativa} \label{cap:analiseResultados}
Esta seção apresenta os critérios definidos para a análise das soluções existentes, a análise comparativa propriamente dita e os resultados obtidos com a análise. A metodologia de comparação entre as as soluções leva em conta documentações públicas disponíveis que possuam informação pertinente.

\subsection{Critérios de Análise} \label{sec:criterios-analise}

Com base nos conceitos apresentados por \cite{wazlawick2012}, que define os atributos de qualidade internos, externos e de uso de produtos de software; e pela autora \cite{wangenheim2017}, que apresenta heurísticas definidas por \cite{nielsen1995} para assegurar que os produtos são usáveis, esta seção tem por objetivo apresentar os critérios que devem ser considerados na avaliação das soluções para busca de dados musicais, de modo a compará-las e facilitar a escolha pela mais apropriada para uma determinada situação.

Neste trabalho foi realizada uma adaptação de heurísticas para permitir uma boa análise da eficiência e adequação funcional das soluções apresentadas, sendo utilizado informações disponibilizadas nas documentações próprias de cada solução comercial e acadêmica. Já para a análise da usabilidade foi realizada uma adaptação do \textit{MATcH Checklist} disponibilizado pelo Grupo de Qualidade de Software (GQS) da UFSC, onde o conjunto de perguntas possui uma escala de resposta com 2 opções: Sim (a solução atende o objetivo) e Não (a solução não atende o objetivo).

Assim sendo, os seguintes critérios e subcritérios são considerados:

\begin{itemize}
    \item Eficiência: Trata da otimização do uso de recursos de tempo e espaço. Espera-se que o sistema seja o mais eficiente possível de acordo com o tipo de problema que ele soluciona:
    \begin{itemize}
        \item \textit{Comportamento em relação ao tempo}: Mede o tempo que o sistema leva para processar suas funções, ou seja, o tempo de reconhecimento/busca de uma música;
        \item \textit{Utilização de recursos}: Avalia a complexidade das estratégias e algoritmos utilizados na recuperação de informação musical;
        \item \textit{Bitrate}: Mede a qualidade do áudio. Essa qualidade consiste no número médio de bits que será comprimido em um segundo de dados. A unidade utilizada é o KBPS ou 1000 BITS por segundo;
    \end{itemize}
    \item Adequação Funcional: Mede o grau no qual o produto oferece funções que satisfazem necessidades estabelecidas e implicadas quando o produto é usado sob condições especificadas:
    \begin{itemize}
        \item \textit{Disponibilidade}: Avalia a disponibilidade da aplicação em diferentes plataformas;
        \item \textit{Modelo de desenvolvimento}: Avalia se a solução é de código aberto, dando a possibilidade para que qualquer um consulte, examine ou modifique o produto;
        \item \textit{Integrações}: Avalia se a solução permite extensões e/ou integrações com outras aplicações;
        \item \textit{Acessibilidade}: Avalia se a solução possui acesso ao acervo de músicas on-line e/ou off-line;
        \item \textit{Busca de dados}: Avalia se a solução foi projetada para \textit{matching} exato ou por similaridade (aproximado);
        \item \textit{Inclusão da dados}: Avalia se a solução permite o envio de músicas feito pelo usuário;
        \item \textit{Modelo de pagamento}: Avalia o custo da solução, como por exemplo: Gratuito, Pago ou Freemium.
    \end{itemize}
    \item Usabilidade: Avalia o grau no qual o produto tem atributos que permitem que seja entendido e que seja atraente ao usuário, quando usado sob condições especificadas:
    \begin{itemize}
        \item \textit{Visibilidade do status do sistema}: O sistema deve sempre manter o usuário informado sobre o que está acontecendo. Por exemplo, os componentes interativos selecionados são claramente distintos dos demais?
        \item \textit{Prevenção de erros}: Mensagens de erros devem ser claras e objetivas, devem indicar o problema com precisão e sugerir uma solução;
        \item \textit{Flexibilidade e eficiência de uso}: A solução deve permitir configuração de ações frequentes. Por exemplo, as funções mais utilizadas são facilmente acessadas?
        \item \textit{Estética e Design minimalista}: Mensagens de diálogos não devem conter informações irrelevantes. Por exemplo, o menu é esteticamente simples e claro, com opções fáceis de encontrar, dispostas em uma ordem lógica e com títulos curtos?
        \item \textit{Pouca interação homem/dispositivo}: Qualquer informação deve ser fácil de pesquisar e deve ser focada na tarefa do usuário. Por exemplo, a navegação da solução é intuitiva, é fácil chegar à tela desejada?
    \end{itemize}
\end{itemize}


\subsection{Análise} \label{sec:analise}

Uma comparação é primeiramente realizada entre as soluções comerciais, seguida das soluções acadêmicas.

\subsubsection{Soluções Comerciais}

Inicialmente, para a análise das soluções comerciais, foram criadas 3 tabelas, uma para cada grande critério analisado para melhor visualização.

Em relação ao critério de Eficiência mostrado na Tabela \ref{tab:eficiencia-comercial}, o comportamento das soluções em relação ao tempo foram constatados as diferenças verificando o tipo de método utilizado para o reconhecimento/busca de músicas.Os testes foram realizados usando conexão de Internet Wi-Fi e Dados Móveis (4G). Com um smartphone e um cronômetro, foram feitas 5 execuções em cada aplicação com músicas diferentes. E então, calculado a média de todos os tempos para cada aplicação.

Para as soluções utilizando RPC, foi testada a busca de músicas pela correspondência de texto através de metadados (ou seja, através da informação do título, álbum ou genero da música). Neste caso, a busca demorou até 30s para retornar uma amostra de resultados. Para as soluções que utilizam \textit{Fingerprint} e fazem reconhecimento de música de forma exata (E), foi testada a busca através da informação de uma parte de áudio "original". Já para as soluções de reconhecimento de músicas de forma aproximada (A), foi testada a busca através da informação de uma parte de áudio "original", do canto ou do cantarolar. Ambas as soluções podem variar de 5s até 13s para retornar uma amostra de resultados.

Cabe observar aqui que uma análise feita apenas sobre o comportamento geral das soluções em relação ao tempo não seria relevante, já que tempos entre 5s e 30s são relativos à situação em que o usuário está no momento. Por exemplo, o usuário pode estar em lugares onde a conexão de Internet não é boa ou quase nula, ou pode estar em lugares onde a conexão de Internet é ótima, já que o reconhecimento de uma música depende de uma conexão de Internet. Quanto melhor a conexão, mais rápido será o reconhecimento da música.

É possível verificar que tanto soluções para o reconhecimento de músicas de forma aproximada, como o MusicID e o ACRCloud, quanto soluções para o reconhecimento de forma exata, como o Shazam, Deezer e Musixmatch, podem utilizar o mesmo método de \textit{Fingerprint}. Apesar de utilizarem o mesmo método, a forma como pode ter sido desenvolvido é que gerou a vantagem competitiva dentre os concorrentes do mesmo ramo.

O SoundHound utiliza a tecnologia de Inteligência Artificial (IA) para o reconhecimento de músicas. O desenvolvimento de uma solução baseada em IA se dá através de métodos e algoritmos de aprendizagem de máquina, como Classificação e Clustering, porém não foi encontrada documentação que especificasse o método ou algoritimo utilizado pelo SoundHound para o reconhecimento das músicas. Com essa solução, é possível recuperar músicas através da voz, cantarolando ou até mesmo informando uma parte do áudio "original". Conforme os usuários cantam e conforme a escolha da música do resultado amostrado, a IA associa a cantoria àquela música. Então, quanto mais cantar e buscar as músicas "certas"\ para a cantoria, a taxa de acerto aumenta, formando uma rede totalmente interligada para o reconhecimento correto de músicas.

Há também as soluções que utilizam o método de Recuperação por Conteúdo (RPC), que seriam as cadeias de caracteres, em buscas, por exemplo, por título, álbum ou gênero da música - o que chamamos de metadados -, conforme já explícitado em seções anteriores deste trabalho. Destas soluções, pode-se citar o Spotify e o SoundCloud. A busca das músicas no Spotify, por exemplo, é feita exclusivamente através de texto, não sendo possível adicionar músicas criadas pelo usuário. Entretanto, é possível criar playlists com as suas músicas preferidas.

Para o Musipedia, não foi encontrada documentação que especificasse o método ou algoritimo utilizado para o reconhecimento das músicas.

Quanto ao critério de Bitrate, das 9 soluções analisadas, foi encontrado essa informação em apenas 4. Diferentes bitrates podem produzir qualidade de som variável. Assim, quanto maior o bitrate, mais vezes por segundo o som original é amostrado, produzindo assim uma reprodução mais autêntica e melhor do som \cite{sandisk}.

O Deezer e Spotify apresentam bitrate de até 320kbps, possuindo 3MB por minuto de áudio e uma qualidade aproximada à de um CD. Já o MusicID e o SoundCloud apresentam bitrate de até 128kbps, possuindo 1MB por minuto de áudio e uma qualidade mínima aceitável em termos de alta fidelidade do áudio \cite{sean2012}.

%%%====== INICIO TABELA EFICIÊNCIA COMERCIAL ======%%%
\begin{table}[!htb]
\centering
\caption{Análise das soluções comerciais: Critério de Eficiência}
\tabcolsep=0.12cm
\label{tab:eficiencia-comercial}
\begin{tabular}{c|l|c|c|c}
\hline
\multicolumn{2}{c|}{\multirow{2}{*}{\textbf{CRITÉRIOS}}} & \multicolumn{3}{|c}{\textbf{Eficiência}} \\ \cline{3-5} 
\multicolumn{2}{c|}{} & \textbf{\begin{tabular}[c]{@{}c@{}}Comportamento\\ em relação ao\\ tempo\end{tabular}} & \textbf{\begin{tabular}[c]{@{}c@{}}Utilização de\\ recursos\end{tabular}} & \textbf{Bitrate} \\ \hline
\multirow{9}{*}{\begin{tabular}[c]{@{}c@{}}C\\ O\\ M\\ E\\ R\\ C\\ I\\ A\\ L\end{tabular}} & \textbf{MusicID} & Até 8s & FP & Até 128kbps \\ \cline{2-5} 
 & \textbf{Shazam} & Até 8s & FP & - \\ \cline{2-5} 
 & \textbf{SoundHound} & Até 5s & IA & - \\ \cline{2-5} 
 & \textbf{Deezer} & Até 10s & RPC e FP & Até 320kbps \\ \cline{2-5} 
 & \textbf{Spotify} & Até 30s & RPC & Até 320kbps \\ \cline{2-5} 
 & \textbf{SoundCloud} & Até 30s & RPC & Até 128kbps \\ \cline{2-5} 
 & \textbf{Musixmatch} & Até 13s & RPC e FP & - \\ \cline{2-5} 
 & \textbf{ACRCloud} & Até 5s & FP & - \\ \cline{2-5} 
 & \textbf{Musipedia} & - & - & - \\ \hline
\end{tabular}
\caption*{Legenda: FP - Fingerprint; IA - Inteligência Artificial; RPC - Recuperação por Conteúdo;}
\end{table}
%%%====== FIM TABELA EFICIÊNCIA COMERCIAL ======%%%

Em relação ao critério de Adequação Funcional mostrado na Tabela \ref{tab:adequacaoFuncional-comercial}, das 9 soluções analisadas, 7 são multiplataformas, sendo em sua maioria voltadas para uso em smartphones e 2 de uso exclusivo via Web.

Pode-se verificar que as soluções comerciais, em sua grande maioria, possuem um modelo de desenvolvimento fechado (F), ou seja, seu código não pode ser alterado, mas disponibilizam APIs Web para a comunidade de desenvolvedores, para que a solução possa ser incorporada a seus próprios sites e aplicações.

Quanto à acessibilidade, todas precisam de conexão com a Internet para o uso de suas funcionalidades e apenas 2 das 9 soluções analisadas possuem acesso off-line, que se dá através do download de músicas para a memória interna do dispositivo. As 2 soluções são Spotify e Deezer.

Das 9 soluções comerciais analisadas, 4 possuem reconhecimento de músicas de forma aproximada (A), ou seja, é possível realizar buscas de músicas por similaridade através da voz, sendo que 2 delas permite cantarolar. Já as 5 demais soluções realizam o reconhecimento de músicas de forma exata (E), sendo necessária uma parte de áudio "original"\ e/ou uma busca por conteúdo através de metadados.

Apenas 3 soluções permitem a inclusão de músicas criadas por usuários, enquanto as outras 6 permitem a inclusão de novas músicas somente através de contatos com gravadoras e/ou artistas. Destas 3 primeiras, duas reconhecem músicas de forma aproximada: ACRCloud e Musipedia. A primeira é um serviço na nuvem, sendo, até o momento, o maior banco de dados de músicas e um serviço utilizado pela maioria das outras soluções aqui analisadas, como Musixmatch, que permite o reconhecimento de músicas por similaridade, e o Musipedia é uma wikipedia de músicas que aceita contribuições musicais de diversas formas: através da voz, partes de músicas, ou até em formato MIDI (quando o som é criado digitalmente). 

Com exceção do ACRCloud, todas as demais soluções possuem versões gratuitas para uso, com a possibilidade de pagar uma mensalidade e não ter interrupções e propagandas entre as músicas. Ainda, poucas soluções possuem integrações com outros serviços e/ou aplicações. Com o Shazam, por exemplo, é possível integrar-se ao Spotify e, então, ao encontrar uma música, poder ouvi-la por completo. Da mesma forma para o SoundHound, além de ser possível o compartilhamento da sua pesquisa com o Twitter. Com o Musixmatch integrado ao Spotify, por exemplo, ao encontrar uma música, você acompanha a música com a letra em tempo real. O ACRCloud e o Musixmatch são as soluções que possuem o maior número de integrações.

%%%====== INICIO TABELA ADEQUAÇÃO FUNCIONAL COMERCIAL ======%%%
\begin{table}[!htb]
\centering
\caption{Análise das soluções comerciais: Critério de Adequação Funcional}
\fontsize{8.5}{12}\selectfont
\tabcolsep=0.12cm
\label{tab:adequacaoFuncional-comercial}
\begin{tabular}{c|l|c|c|c|c|c|c|c}
\hline
\multicolumn{2}{c|}{\multirow{2}{*}{\textbf{CRITÉRIOS}}} & \multicolumn{7}{|c}{\textbf{Adequação funcional}} \\ \cline{3-9} 
\multicolumn{2}{c|}{} & \textbf{Disponibilidade} & \textbf{\begin{tabular}[c]{@{}c@{}}Modelo\\ de\\ desenv.\end{tabular}} & \textbf{Integrações} & \textbf{Acessibilidade} & \textbf{\begin{tabular}[c]{@{}c@{}}Busca\\ de\\ dados\end{tabular}} & \textbf{\begin{tabular}[c]{@{}c@{}}Inclusão\\ de\\ dados\end{tabular}} & \textbf{\begin{tabular}[c]{@{}c@{}}Modelo\\ de\\ Pagamento\end{tabular}} \\ \hline
\multirow{9}{*}{\begin{tabular}[c]{@{}c@{}}C\\ O\\ M\\ E\\ R\\ C\\ I\\ A\\ L\end{tabular}} & \textbf{MusicID} & iOS, Android & F & não & On & A & não & G \\ \cline{2-9} 
 & \textbf{Shazam} & iOS, Android & F & \begin{tabular}[c]{@{}c@{}}Spotify, \\ Google Music, \\ Apple Music\end{tabular} & On & E & não & G \\ \cline{2-9} 
 & \textbf{SoundHound} & iOS, Android & F & \begin{tabular}[c]{@{}c@{}}Spotify,\\ Youtube,\\ Twitter\end{tabular} & On & A & não & F \\ \cline{2-9} 
 & \textbf{Deezer} & \begin{tabular}[c]{@{}c@{}}iOS, Android, \\ Windows, Web\end{tabular} & F & \begin{tabular}[c]{@{}c@{}}Google Music,\\ Twitter, \\ Facebook\end{tabular} & On/Off & E & não & F \\ \cline{2-9} 
 & \textbf{Spotify} & \begin{tabular}[c]{@{}c@{}}iOS, Android, \\ Windows, Web\\ Linux, OS X\end{tabular} & F & não & On/Off & E & não & F \\ \cline{2-9} 
 & \textbf{SoundCloud} & iOS, Android & F & não & On & E & sim & F \\ \cline{2-9} 
 & \textbf{Musixmatch} & iOS, Android & F & \begin{tabular}[c]{@{}c@{}}Spotify,\\ Deezer,\\ Google Music, \\ Youtube,\\ Twitter, \\ Facebook, \\ Last.fm\end{tabular} & On & E & não & F \\ \cline{2-9} 
 & \textbf{ACRCloud} & Web & F & \begin{tabular}[c]{@{}c@{}}Spotify, \\ Deezer, \\ Youtube, \\ ISRC,  \\ UPC, \\ LyricFind, \\ Music Story, \\ SyncPower\end{tabular} & On & A & sim & P \\ \cline{2-9} 
 & \textbf{Musipedia} & Web & A & não & On & A & sim & G \\ \hline
\end{tabular}
\caption*{Legenda: F - Fechado; A - Aberto; E - Busca Exata; A - Busca Aproximada; G - Gratuito; F - Freemiun; P - Premium;}
\end{table}
%%%====== FIM TABELA ADEQUAÇÃO FUNCIONAL COMERCIAL ======%%%

Com relação à usabilidade mostrado na Tabela \ref{tab:usabilidade-comercial}, testes foram realizados considerando os critérios e subcritérios descritos na seção \ref{sec:criterios-analise}. Foi realizada uma adaptação do \textit{MATcH Checklist} disponibilizado pelo Grupo de Qualidade de Software (GQS) da UFSC, onde o conjunto de perguntas possui uma escala de resposta com 2 opções: Sim (a solução atende o objetivo) e Não (a solução não atende o objetivo). Cada solução possui particularidades que podem atrair ou afastar o usuário. Os resultados dos testes são discriminados abaixo, feitos em um smartphone, modelo Asus Zenfone 4, com sistema operacional Android.

Todas as soluções são simples, possuem navegação intuitiva e são de fácil execução. As funções mais utilizadas são facilmente acessadas. Os componentes interativos são claramente distintos uns dos outros, com ícones intuitivos. Com exceção do Musipedia, as soluções possuem uma linguagem clara, concisa e funcionam corretamente, não apresentando problemas. Com relação à Musipedia (ver Figura \ref{fig:musipedia}), ela é uma enciclopédia de músicas e sua navagação não é muito intuitiva, além de possuir textos longos, não apresentando uma linguagem clara e concisa. Não funciona corretamente, não apresentando mensagens de erros claras e objetivas.

No caso do SoundHound, Deezer e Spotify, o menu é esteticamente simples, mas suas versões gratuitas possuem propagandas, o que polui a tela. Ao se adquirir a versão paga, as propagandas são retiradas. Já o MusicID e o Musixmatch não possuem muitas funcionalidades e o menu é esteticamente simples, claro e sem propagandas.

Deezer, Spotify e SoundCloud executam buscas de músicas pela correspondência de texto (metadados) para identificar arquivos de música digital, ou seja, através da informação do título, álbum ou genero da música.

O MusicID (ver Figura \ref{fig:musicID}) é de uso exclusivo para o reconhecimento de músicas ouvindo você cantar, cantarolar ou que são tocadas ao seu redor. O mesmo vale para o SoundHound (ver Figura \ref{fig:soundHound}), com o diferencial de possuir mais funcionalidades além do reconhecimento de músicas, como a possibilidade de criar playlists e o compartilhamento com o Twitter. Além das informações básicas sobre a música, o aplicativo sugere alguns vídeos que podem ser assistidos diretamente no SoundHound.

O Shazam (ver Figura \ref{fig:shazam}) é similar ao MusicID e ao SoundHound. A diferença é que faz o reconhecimento de músicas apenas tocadas ao seu redor. Ele também possui a funcionalidade de tirar fotos de QRCodes para realizar a busca da música. O Deezer (ver Figura \ref{fig:deezer}) é similar ao Shazam, pois também possui a funcionalidade de reconhecimento de músicas que são tocadas ao seu redor e a possibilidade do download das músicas para uso off-line, além da criação de playlists.

O Spotify (ver Figura \ref{fig:spotify}) é similar ao Deezer em termos de funcionalidades. A diferença é que o Spotify não possui reconhecimento das músicas pelo som (trecho de áudio ou voz). Para usufruir de todas as possibilidades do aplicativo, como o download de músicas para uso off-line, é necessário obter a versão premium. Por sua vez, o SoundCloud (ver Figura \ref{fig:soundcloud}) é similar ao Spotify, com a diferença de ser o único aplicativo comercial analisado que permite a inclusão de músicas criadas pelos usuários.

O Musixmatch (ver Figura \ref{fig:musixmatch}) é um aplicativo que sincroniza letras de músicas e também permite o reconhecimento de músicas que são tocadas ao seu redor. Ainda, o ACRCloud (ver Figura \ref{fig:acrcloud}) é um serviço na nuvem, com um grande banco de dados musical. Além de oferecer o serviço de reconhecimento de músicas, ele também permite o reconhecimento de músicas pela web, como um teste do seu serviço.

%%%====== INICIO TABELA USABILIDADE COMERCIAL ======%%%
\begin{table}[!htb]
\centering
\caption{Análise das soluções comerciais: Critério de Usabilidade}
\fontsize{9}{12}\selectfont
\tabcolsep=0.12cm
\label{tab:usabilidade-comercial}
\begin{tabular}{c|l|c|c|c|c|c}
\hline
\multicolumn{2}{c|}{\multirow{2}{*}{\textbf{CRITÉRIOS}}} & \multicolumn{5}{c}{\textbf{Usabilidade}} \\ \cline{3-7} 
\multicolumn{2}{c|}{} & \textbf{\begin{tabular}[c]{@{}c@{}}Visibilidade \\ do status \\ do sistema\end{tabular}} & \textbf{\begin{tabular}[c]{@{}c@{}}Prevenção\\ de erros\end{tabular}} & \textbf{\begin{tabular}[c]{@{}c@{}}Flexibilidade \\ e eficiência \\ de uso\end{tabular}} & \textbf{\begin{tabular}[c]{@{}c@{}}Estética e \\ Design\\  minimalista\end{tabular}} & \textbf{\begin{tabular}[c]{@{}c@{}}Pouca interação\\  homem/dispositivo\end{tabular}} \\ \hline
\multirow{9}{*}{\begin{tabular}[c]{@{}c@{}}C\\ O\\ M\\ E\\ R\\ C\\ I\\ A\\ L\end{tabular}} & \textbf{MusicID} & sim & sim & sim & sim & sim \\ \cline{2-7} 
 & \textbf{Shazam} & sim & sim & sim & sim & sim \\ \cline{2-7} 
 & \textbf{SoundHound} & sim & sim & sim & sim & sim \\ \cline{2-7} 
 & \textbf{Deezer} & sim & sim & sim & sim & sim \\ \cline{2-7} 
 & \textbf{Spotify} & sim & sim & sim & sim & sim \\ \cline{2-7} 
 & \textbf{SoundCloud} & sim & sim & sim & sim & sim \\ \cline{2-7} 
 & \textbf{Musixmatch} & sim & sim & sim & sim & sim \\ \cline{2-7} 
 & \textbf{ACRCloud} & sim & sim & sim & sim & sim \\ \cline{2-7} 
 & \textbf{Musipedia} & sim & não & sim & não & sim \\ \hline
\end{tabular}
\end{table}
%%%====== FIM TABELA USABILIDADE COMERCIAL ======%%%

\subsubsection{Soluções Acadêmicas}

Em relação ao critério de Eficiência mostrado na Tabela \ref{tab:eficiencia-academico}, o comportamento das soluções em relação ao tempo foi encontrada documentação apenas para o Tunebot. A solução utiliza QBH e, segundo \cite{tunebot}, o comprimento mediano de uma música cantada por um usuário é de cerca de 18 segundos de áudio, levando cerca de 5s para retornar resultados a partir do momento em que a aplicação recebe a música. Para comparação, a música mais longa recebida até o momento tem cerca de 48 segundos e a aplicação demora cerca de 13s para retornar uma resposta. O tempo de retorno é uma função de vários fatores, incluindo o tamanho do banco de dados e o comprimento da consulta, tanto em termos da duração total do áudio quanto do número de notas que o usuário cantou.

Cada solução possui ferramentas variadas para uso distinto no reconhecimento de músicas, o que gera diversas formas de utilização dos recursos. Por exemplo, o MusicMiner (ver Figura \ref{fig:musicminer}) possui ferramentas que utilizam os métodos de Classificação e Visualização através de Mapas Musicais Topográficos.

O AMUSE (ver Figura \ref{fig:amuse}) e o Java MIR (ver Figura \ref{fig:jmir}) utilizam o método de Classificação. Já o Tunebot (ver Figura \ref{fig:tunebotWeb} e \ref{fig:tunebotIphone}) utiliza Query by Humming (QBH). O CLAM (ver Figura \ref{fig:clam}) utiliza Spectral Modeling Synthesis (SMS), e o MIRtoolbox (ver Figura \ref{fig:mirtoolbox}) é um acoplado de ferramentas baseado em técnicas de mineração de dados. Por ser um pacote de ferramentas, são variadas técnicas para recuperação da informação musical que ele oferece, como por exemplo, clusterização e classificação.

Quanto ao critério de bitrate, das 6 soluções analisadas, não foi encontrada documentação que especificasse o bitrate utilizado para o fluxo de transferência das músicas.

%%%====== INICIO TABELA EFICIÊNCIA ACADÊMICO ======%%%
\begin{table}[!htb]
\centering
\caption{Análise das soluções acadêmicas: Critério de Eficiência}
\tabcolsep=0.12cm
\label{tab:eficiencia-academico}
\begin{tabular}{c|l|c|c|c}
\hline
\multicolumn{2}{c|}{\multirow{2}{*}{\textbf{CRITÉRIOS}}} & \multicolumn{3}{|c}{\textbf{Eficiência}} \\ \cline{3-5} 
\multicolumn{2}{c|}{} & \textbf{\begin{tabular}[c]{@{}c@{}}Comportamento\\ em relação ao \\ tempo\end{tabular}} & \textbf{\begin{tabular}[c]{@{}c@{}}Utilização de\\ recursos\end{tabular}} & \textbf{Bitrate} \\ \hline
\multirow{6}{*}{\begin{tabular}[c]{@{}c@{}}A\\ C\\ A\\ D\\ Ê\end{tabular}} & \textbf{MusicMiner} & - & C e V & - \\ \cline{2-5} 
 & \textbf{CLAM} & - & SMS & - \\ \cline{2-5} 
 & \textbf{MIRtoolbox} & - & Diversas & - \\ \cline{2-5} 
 & \textbf{AMUSE} & - & C & - \\ \cline{2-5} 
 & \textbf{Java MIR} & - & C & - \\ \cline{2-5} 
 & \textbf{Tunebot} & Até 13s & QBH & - \\ \hline
\end{tabular}
\caption*{Legenda: C - Classificação; V - Visualização; SMS - Spectral Modeling Synthesis; QBH - Query by Humming;}
\end{table}
%%%====== FIM TABELA EFICIÊNCIA ACADÊMICO ======%%%

Em relação ao critério de Adequação Funcional mostrado na Tabela \ref{tab:adequacaoFuncional-academico}, as soluções acadêmicas apresentam um modelo de desenvolvimento aberto, ou seja, seu código pode ser alterado por outros usuários, contribuindo para a resolução de problemas existentes nas aplicações. Das 6 soluções analisadas, todas possuem reconhecimento de músicas de forma aproximada (A), ou seja, permitem realizar buscas de músicas através de trechos de áudio e/ou voz e/ou codificação MIDI e/ou por conteúdo (metadados).

Todas as soluções permitem a inclusão de músicas criadas por usuários. Quanto à disponibilidade, com exceção do Tunebot, todas as soluções estão disponíveis para Linux, Mac OSX e Windows e seu acesso é off-line, ou seja, é necessário o download e instalação do projeto no dispositivo. O download e instalação das soluções é feito de forma gratuita. O Tunebot está disponivel na Web e também possui uma versão beta para smartphones iOS, tendo o seu acesso on-line (Ver Figura \ref{fig:tunebotIphone}).

O AMUSE e o Tunebot são as duas soluções que possuem integração com outras aplicações. O AMUSE faz parte do rol de ferramentas do MIRtoolbox. Já o Tunebot tem integração com o \textit{Karaoke Callout}\footnote{\textit{Karaoke Callout} é um jogo de karaokê para a plataforma iOS da Apple que permite aos usuários "tunebot" desafiarem uns aos outros para um duelo de canto \cite{karaokeCallout2006}.}, que ajuda a construir a base de conhecimento de músicas do Tunebot.

%%%====== INICIO TABELA ADEQUAÇÃO FUNCIONAL ACADÊMICO ======%%%
\begin{table}[!htb]
\centering
\caption{Análise das soluções acadêmicas: Critério de Adequação Funcional}
\fontsize{8.5}{12}\selectfont
\tabcolsep=0.12cm
\label{tab:adequacaoFuncional-academico}
\begin{tabular}{c|l|c|c|c|c|c|c|c}
\hline
\multicolumn{2}{c|}{\multirow{2}{*}{\textbf{CRITÉRIOS}}} & \multicolumn{7}{|c}{\textbf{Adequação funcional}} \\ \cline{3-9} 
\multicolumn{2}{c|}{} & \textbf{Disponibilidade} & \textbf{\begin{tabular}[c]{@{}c@{}}Modelo\\ de\\ desenv.\end{tabular}} & \textbf{Integrações} & \textbf{Acessibilidade} & \textbf{\begin{tabular}[c]{@{}c@{}}Busca\\ de\\ dados\end{tabular}} & \textbf{\begin{tabular}[c]{@{}c@{}}Inclusão\\ de\\ dados\end{tabular}} & \textbf{\begin{tabular}[c]{@{}c@{}}Modelo\\ de\\ Pagamento\end{tabular}} \\ \hline
\multirow{6}{*}{\begin{tabular}[c]{@{}c@{}}A\\ C\\ A\\ D\\ Ê\\ M\\ I\\ C\\ O\end{tabular}} & \textbf{MusicMiner} & \begin{tabular}[c]{@{}c@{}}Linux, Mac OSX,\\ Windows\end{tabular} & A & não & Off & A & sim & G \\ \cline{2-9} 
 & \textbf{CLAM} & \begin{tabular}[c]{@{}c@{}}Linux, Mac OSX,\\ Windows\end{tabular} & A & não & Off & A & sim & G \\ \cline{2-9} 
 & \textbf{MIRtoolbox} & \begin{tabular}[c]{@{}c@{}}Linux, Mac OSX,\\ Windows\end{tabular} & A & não & Off & A & sim & G \\ \cline{2-9} 
 & \textbf{AMUSE} & - & A & MIRtoolbox & Off & A & sim & G \\ \cline{2-9} 
 & \textbf{Java MIR} & \begin{tabular}[c]{@{}c@{}}Linux, Mac OSX,\\ Windows\end{tabular} & A & não & Off & A & sim & G \\ \cline{2-9} 
 & \textbf{Tunebot} & Web, iOS & A & \begin{tabular}[c]{@{}c@{}}Karaokê\\ Callout\end{tabular} & On & A & sim & G \\ \hline
\end{tabular}
\caption*{Legenda: F - Fechado; A - Aberto; E - Busca Exata; A - Busca Aproximada; G - Gratuito; F - Freemiun; P - Premium;}
\end{table}
%%%====== FIM TABELA ADEQUAÇÃO FUNCIONAL ACADÊMICO ======%%%

Como explicitado anteriormente, as soluções analisadas tem seu acesso off-line. Diante disso, para o funcionamento correto de cada aplicação é necessário a instalação de projetos de terceiros. Desta forma, com a impossibilidade de funcionamento das aplicações, não foi possível analisar os critérios de usabilidade das soluções acadêmicas.

\subsection{Discussão dos Resultados} \label{sec:resultados}

Conforme explicado na seção anterior, a análise de solucões foi dividida em dois grupos: soluções comerciais e soluções acadêmicas. Uma análise entre todas elas seria inviável, pois cada grupo se difere no motivo de uso e é voltado a públicos-alvo diferentes. As soluções comerciais são mais utilizadas por usuários finais. Já as soluções acadêmicas são voltadas a usuários pesquisadores da comunidade de MIR.

\subsubsection{Soluções Comerciais}

Com relação à análise das soluções comerciais, pode-se concluir que as soluções focam na usabilidade visando facilitar o uso para o seu público-alvo, que apenas as utilizam no seu dia-a-dia. Estes usuários não possuem muito interesse em como essas soluções funcionam, desejando apenas que a busca seja rápida e satisfaça os filtros informados.

Os testes realizados de reconhecimento e busca de músicas, por metadados ou por som, demonstraram que as aplicações atenderam bem a todas as funcionalidades oferecidas. Se o objetivo for o reconhecimento de músicas quando não se conhece nenhum valor de metadado, o Shazam e o Deezer são essenciais. Porém, se o objetivo for aquela música que não sai da cabeça e precisar cantar ou cantarolar, então o ideal é usar o SoundHound ou o MusicID. Se o conhecimento da letra da música for indispensável, é possível usar o Musixmatch.

Se o objetivo for playlists e download de músicas, para não ter a necessidade do uso de conexão com a Internet, o Spotify ou o Deezer atendem a demanda. Já o Musipedia, em particular, é uma solução voltada a usuários músicos, que compõem músicas através da codificação MIDI.

Em relação ao tempo, o ideal é a conexão de Internet Wi-Fi. Nestas condições, todas as soluções, exceto o Musipedia, fazem o reconhecimento/busca de músicas no tempo mínimo de 5s. Nesse caso, saber o objetivo do usuário com relação a desempenho é imprescindível para a escolha de qual solução usar.

Pode-se concluir que o método mais utilizado para o reconhecimento de músicas é o \textit{Fingerprint}. Inclusive, o uso do serviço do ACRCloud tem sido incorporado em cada vez mais soluções para o reconhecimento das músicas, como é o caso do Musixmatch e do Deezer. Outro método bastante utilizado é o RPC, principalmente pelas soluções que permitem apenas buscas exatas, onde é feita a correspondência de texto para identificar arquivos de música digital, ou seja, através da informação do título, álbum ou genero da música.


\subsubsection{Soluções Acadêmicas}

Com relação à análise realizada para as soluções acadêmicas, pode-se concluir que as soluções focam na otimização e desempenho no que diz respeito ao reconhecimento de músicas. O desenvolvimento de tais aplicações é voltado para os usuários pesquisadores da comunidade de MIR, com o objetivo de trazer inovação na busca de dados musicais.

Todas as soluções incluem uma estratégia para busca por similaridade (a maioria delas utilizando técnicas de Classificação), são gratuitas e permitem a inclusão de dados musicais em seus repositórios. Ainda, a maioria são soluções off-line e estão disponíveis em diversas plataformas. Por outro lado, poucas soluções oferecem integração com outras soluções ou aplicações que tocam músicas. Inclusive apresentam um modelo de desenvolvimento aberto, ou seja, seu código pode ser alterado por outros usuários.

Houve dificuldade no uso das soluções acadêmicas e por este motivo, não houve a análise dos critérios de usabilidade, além de não ter sido possível o teste prático dos critérios de eficiência de desempenho, o que poderia ter enriquecido ainda mais a presente pesquisa. A documentação existente para cada proposta sugere que o desenvolvimento das soluções acadêmicas geralmente tem o objetivo de validar uma determinada pesquisa, sendo após oferecida gratuitamente à comunidade para futuras pesquisas e o aprimoramento da solução. Desta forma, a usabilidade nem sempre é o foco principal. 

Por outro lado, certos processos já foram desenvolvidos e estão sendo utilizados como parte de uma nova solução, como é o caso do projeto AMUSE, que faz parte do rol de ferramentas do MIRtoolbox. Outro caso é o Tunebot, que possui uma versão beta desenvolvida para plataformas iOS, sendo um dos seus objetivos ajudar pesquisadores na área de reconhecimento de músicas que utilizam o algoritmo \textit{Query by Humming}, facilitando uma pesquisa mais precisa do desempenho do mundo real do que seria possível com conjuntos de dados existentes.

Por fim, a escolha de um método para a recuperação da informação musical vai depender do objetivo da solução a ser desenvolvida. Neste sentido, quanto mais aprimorados forem os métodos e algoritmos de soluções acadêmicas, melhores soluções comerciais serão desenvolvidas para a utilização pelo usuário final.
