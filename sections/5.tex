\section{Considerações Finais} \label{cap:consideracoes-finais}

Este trabalho teve como objetivo analisar soluções para a busca por similaridade de dados musicais, ou seja, soluções que não necessariamente buscam dados musicais apenas através do casamento direto de parâmetros de entrada para a busca, como título da música, palavras-chave ou um áudio como parte da música, mas tambem atavés do casamento aproximado (ou similar) destes parâmetros. Além de uma pesquisa aplicada sobre o que são dados musicais, sobre a forma como os dados são tratados e armazenados, para que possam posteriormente ser recuperados e ouvidos no dia-a-dia, este trabalho buscou reunir e comparar informações com o propósito de contribuir com futuros trabalhos que desejam desenvolver soluções para a busca por similaridade de dados musicais.

Pela observação das características analisadas nas Tabelas \ref{tab:eficiencia-comercial}, \ref{tab:adequacaoFuncional-comercial} e \ref{tab:usabilidade-comercial}, as soluções comerciais enfatizam a usabilidade, visando facilitar o uso para o público-alvo, que é o usuário final no seu dia-a-dia. O objetivo destas soluções pode ser o reconhecimento de músicas através de uma parte da música ou da voz, bem como a criação de playlists e download de músicas para uso off-line. O método mais utilizado para o reconhecimento de músicas é o Fingerprint.

As soluções acadêmicas analisadas nas Tabelas \ref{tab:eficiencia-academico} e \ref{tab:adequacaoFuncional-academico}, por sua vez, enfatizam o desempenho no reconhecimento de músicas. O desenvolvimento de tais soluções é destinado principalmente a usuários pesquisadores da comunidade de MIR com o objetivo de trazer inovação na maneira como são recuperados os dados musicais. Todas as propostas incluem uma estratégia para busca por similaridade (a maioria delas utilizando técnicas de Classificação), são gratuitas e permitem a inclusão de dados musiciais em seus repositórios. Ainda, a maioria são soluções off-line e estão disponíveis em diversas plataformas. Por outro lado, poucas soluções oferecem integração com outras soluções ou aplicações que tocam músicas. 

Como trabalhos futuros, sugere-se, primeiramente, a aquisição dos protótipos das soluções acadêmicas para que uma avaliação de usabilidade destas propostas possa ser realizada. Neste trabalho, devido a restrições de tempo, não foi possível ter acesso a essas soluções. Outra sugestão é um aprofundamento no estudo dos métodos e algoritmos utilizados para a recuperação da informação musical, como \textit{FingerPrint}, Recuperação por Conteúdo e Classificação, bem como uma comparação para verificar os métodos mais eficientes. Por fim, o desenvolvimento e avaliação de soluções para recuperação de informação musical poderia contar com a participação de mais mulheres no programa de mentoria WIMIR\footnote{https://wimir.wordpress.com/mentoring-program/}, incluindo soluções para busca por similaridade.